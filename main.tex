\documentclass{article}
\usepackage[russian]{babel}

\begin{document}

\section{Нормализация}
   $
(\lambda a.(\lambda b.bb)(\lambda b.bb))b((\lambda c.(cb))(\lambda a.a)) \to \\
\text{$\beta$-редукция} \\
((\lambda b.bb)(\lambda b.bb))((\lambda c.(cb))(\lambda a.a)) \to \\
\text{$\beta$-редукция, бесконечная редукция, норамльной формы не будет} \\
((\lambda b.bb)(\lambda b.bb))((\lambda c.(cb))(\lambda a.a)) \to \\
\text{$\beta$-редукция} \\
((\lambda b.bb)(\lambda b.bb))((\lambda a.a)b) \to \\
\text{$\beta$-редукция, тождественная функция} \\
((\lambda b.bb)(\lambda b.bb))b \to \\
\text{слева бесконечная редукция, терм воспроизводит самого себя --- нет нормальной формы}
$
слева бесконечная редукция, терм воспроизводит самого себя --- нет нормальной формы



\section{S K K}

\begin{align*}
S &= \lambda x.\lambda y.\lambda z.x z (y z) \\
K &= \lambda x.\lambda y.x \\
I &= \lambda x.x
\end{align*}

Докажем 

\begin{align*}
S K K &= (\lambda x.\lambda y.\lambda z.x z (y z)) K K \\
&\to_\beta (\lambda y.\lambda z.K z (y z)) K \\
&\to_\beta \lambda z.K z (K z) \\
&= \lambda z.(\lambda x.\lambda y.x) z (K z) \\
&\to_\beta \lambda z.(\lambda y.z) (K z) \\
&\to_\beta \lambda z.z \\
&= I
\end{align*}

\end{document}